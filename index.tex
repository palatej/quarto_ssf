% Options for packages loaded elsewhere
\PassOptionsToPackage{unicode}{hyperref}
\PassOptionsToPackage{hyphens}{url}
\PassOptionsToPackage{dvipsnames,svgnames,x11names}{xcolor}
%
\documentclass[
  letterpaper,
  DIV=11,
  numbers=noendperiod]{scrreprt}

\usepackage{amsmath,amssymb}
\usepackage{iftex}
\ifPDFTeX
  \usepackage[T1]{fontenc}
  \usepackage[utf8]{inputenc}
  \usepackage{textcomp} % provide euro and other symbols
\else % if luatex or xetex
  \usepackage{unicode-math}
  \defaultfontfeatures{Scale=MatchLowercase}
  \defaultfontfeatures[\rmfamily]{Ligatures=TeX,Scale=1}
\fi
\usepackage{lmodern}
\ifPDFTeX\else  
    % xetex/luatex font selection
\fi
% Use upquote if available, for straight quotes in verbatim environments
\IfFileExists{upquote.sty}{\usepackage{upquote}}{}
\IfFileExists{microtype.sty}{% use microtype if available
  \usepackage[]{microtype}
  \UseMicrotypeSet[protrusion]{basicmath} % disable protrusion for tt fonts
}{}
\makeatletter
\@ifundefined{KOMAClassName}{% if non-KOMA class
  \IfFileExists{parskip.sty}{%
    \usepackage{parskip}
  }{% else
    \setlength{\parindent}{0pt}
    \setlength{\parskip}{6pt plus 2pt minus 1pt}}
}{% if KOMA class
  \KOMAoptions{parskip=half}}
\makeatother
\usepackage{xcolor}
\setlength{\emergencystretch}{3em} % prevent overfull lines
\setcounter{secnumdepth}{5}
% Make \paragraph and \subparagraph free-standing
\makeatletter
\ifx\paragraph\undefined\else
  \let\oldparagraph\paragraph
  \renewcommand{\paragraph}{
    \@ifstar
      \xxxParagraphStar
      \xxxParagraphNoStar
  }
  \newcommand{\xxxParagraphStar}[1]{\oldparagraph*{#1}\mbox{}}
  \newcommand{\xxxParagraphNoStar}[1]{\oldparagraph{#1}\mbox{}}
\fi
\ifx\subparagraph\undefined\else
  \let\oldsubparagraph\subparagraph
  \renewcommand{\subparagraph}{
    \@ifstar
      \xxxSubParagraphStar
      \xxxSubParagraphNoStar
  }
  \newcommand{\xxxSubParagraphStar}[1]{\oldsubparagraph*{#1}\mbox{}}
  \newcommand{\xxxSubParagraphNoStar}[1]{\oldsubparagraph{#1}\mbox{}}
\fi
\makeatother


\providecommand{\tightlist}{%
  \setlength{\itemsep}{0pt}\setlength{\parskip}{0pt}}\usepackage{longtable,booktabs,array}
\usepackage{calc} % for calculating minipage widths
% Correct order of tables after \paragraph or \subparagraph
\usepackage{etoolbox}
\makeatletter
\patchcmd\longtable{\par}{\if@noskipsec\mbox{}\fi\par}{}{}
\makeatother
% Allow footnotes in longtable head/foot
\IfFileExists{footnotehyper.sty}{\usepackage{footnotehyper}}{\usepackage{footnote}}
\makesavenoteenv{longtable}
\usepackage{graphicx}
\makeatletter
\def\maxwidth{\ifdim\Gin@nat@width>\linewidth\linewidth\else\Gin@nat@width\fi}
\def\maxheight{\ifdim\Gin@nat@height>\textheight\textheight\else\Gin@nat@height\fi}
\makeatother
% Scale images if necessary, so that they will not overflow the page
% margins by default, and it is still possible to overwrite the defaults
% using explicit options in \includegraphics[width, height, ...]{}
\setkeys{Gin}{width=\maxwidth,height=\maxheight,keepaspectratio}
% Set default figure placement to htbp
\makeatletter
\def\fps@figure{htbp}
\makeatother
% definitions for citeproc citations
\NewDocumentCommand\citeproctext{}{}
\NewDocumentCommand\citeproc{mm}{%
  \begingroup\def\citeproctext{#2}\cite{#1}\endgroup}
\makeatletter
 % allow citations to break across lines
 \let\@cite@ofmt\@firstofone
 % avoid brackets around text for \cite:
 \def\@biblabel#1{}
 \def\@cite#1#2{{#1\if@tempswa , #2\fi}}
\makeatother
\newlength{\cslhangindent}
\setlength{\cslhangindent}{1.5em}
\newlength{\csllabelwidth}
\setlength{\csllabelwidth}{3em}
\newenvironment{CSLReferences}[2] % #1 hanging-indent, #2 entry-spacing
 {\begin{list}{}{%
  \setlength{\itemindent}{0pt}
  \setlength{\leftmargin}{0pt}
  \setlength{\parsep}{0pt}
  % turn on hanging indent if param 1 is 1
  \ifodd #1
   \setlength{\leftmargin}{\cslhangindent}
   \setlength{\itemindent}{-1\cslhangindent}
  \fi
  % set entry spacing
  \setlength{\itemsep}{#2\baselineskip}}}
 {\end{list}}
\usepackage{calc}
\newcommand{\CSLBlock}[1]{\hfill\break\parbox[t]{\linewidth}{\strut\ignorespaces#1\strut}}
\newcommand{\CSLLeftMargin}[1]{\parbox[t]{\csllabelwidth}{\strut#1\strut}}
\newcommand{\CSLRightInline}[1]{\parbox[t]{\linewidth - \csllabelwidth}{\strut#1\strut}}
\newcommand{\CSLIndent}[1]{\hspace{\cslhangindent}#1}

\KOMAoption{captions}{tableheading}
\makeatletter
\@ifpackageloaded{bookmark}{}{\usepackage{bookmark}}
\makeatother
\makeatletter
\@ifpackageloaded{caption}{}{\usepackage{caption}}
\AtBeginDocument{%
\ifdefined\contentsname
  \renewcommand*\contentsname{Table of contents}
\else
  \newcommand\contentsname{Table of contents}
\fi
\ifdefined\listfigurename
  \renewcommand*\listfigurename{List of Figures}
\else
  \newcommand\listfigurename{List of Figures}
\fi
\ifdefined\listtablename
  \renewcommand*\listtablename{List of Tables}
\else
  \newcommand\listtablename{List of Tables}
\fi
\ifdefined\figurename
  \renewcommand*\figurename{Figure}
\else
  \newcommand\figurename{Figure}
\fi
\ifdefined\tablename
  \renewcommand*\tablename{Table}
\else
  \newcommand\tablename{Table}
\fi
}
\@ifpackageloaded{float}{}{\usepackage{float}}
\floatstyle{ruled}
\@ifundefined{c@chapter}{\newfloat{codelisting}{h}{lop}}{\newfloat{codelisting}{h}{lop}[chapter]}
\floatname{codelisting}{Listing}
\newcommand*\listoflistings{\listof{codelisting}{List of Listings}}
\makeatother
\makeatletter
\makeatother
\makeatletter
\@ifpackageloaded{caption}{}{\usepackage{caption}}
\@ifpackageloaded{subcaption}{}{\usepackage{subcaption}}
\makeatother

\ifLuaTeX
  \usepackage{selnolig}  % disable illegal ligatures
\fi
\usepackage{bookmark}

\IfFileExists{xurl.sty}{\usepackage{xurl}}{} % add URL line breaks if available
\urlstyle{same} % disable monospaced font for URLs
\hypersetup{
  pdftitle={Quarto-SSF},
  pdfauthor={Jean Palate},
  colorlinks=true,
  linkcolor={blue},
  filecolor={Maroon},
  citecolor={Blue},
  urlcolor={Blue},
  pdfcreator={LaTeX via pandoc}}


\title{Quarto-SSF}
\author{Jean Palate}
\date{2026-02-01}

\begin{document}
\maketitle

\renewcommand*\contentsname{Table of contents}
{
\hypersetup{linkcolor=}
\setcounter{tocdepth}{2}
\tableofcontents
}

\bookmarksetup{startatroot}

\chapter*{Index}\label{index}
\addcontentsline{toc}{chapter}{Index}

\markboth{Index}{Index}

\bookmarksetup{startatroot}

\chapter{}\label{section}

\bookmarksetup{startatroot}

\chapter*{References}\label{references}
\addcontentsline{toc}{chapter}{References}

\markboth{References}{References}

\phantomsection\label{refs}
\begin{CSLReferences}{0}{1}
\end{CSLReferences}

\part{Introduction}

\chapter{Overview of the state space framework of
JD+}\label{overview-of-the-state-space-framework-of-jd}

\section{Description of the general
model}\label{description-of-the-general-model}

The \texttt{general\ linear\ gaussian} state-space model can be written
in many different ways. The form considered in JD+ (\(\ge 3.0\)) is
presented below:

\[ y_t = Z_t \alpha_t + \epsilon_t,\quad \epsilon_t \sim N\left(0, \sigma^2 H_t\right),\quad t> 0\]

\[ \alpha_{t+1} = T_t \alpha_t + \mu_t, \quad \mu_t \sim N \left(0, \sigma^2 V_t \right),\quad t \ge 0 \]

\(y_t\) is the observation at period t, \(\alpha_t\) is the state
vector. \(\epsilon_t, \mu_t\) are assumed to be serially independent at
all leads and lags and independent from each other.\\
In the case of multi-variate models, \(y_t\) is a vector of
observations. However, in most cases, we will use the univariate
approach by considering the observations one by one (univariate handling
of multi-variate models).

The innovations of the state equation will be modelled as

\[ \mu_t = S_t \xi_t, \quad \xi_t \sim N\left( 0, \sigma^2 I\right) \]

In other words, \(V_t=S_t S_t'\)

The initial (\(\equiv t=0\)) conditions of the filter are defined as
follows:

\[ \alpha_{0} = a_{0} + B\delta + \mu_{0}, \quad \delta \sim N\left(0, \kappa I \right),\: \mu_{0} \sim N\left(0, P_*\right)\]

where \(\kappa\) is arbitrary large. \(P_*\) is the variance of the
stationary part of the initial state vector and \(B\) models the diffuse
part. We write \(BB'=P_\infty\).

The definition used in JD+ is quasi-identical to that of Durbin and
Koopman{[}1{]}.

In summary, the model is completely defined by the following quantities
(possible default values are indicated in brackets):

\[ \mathbf{Z_t}, \mathbf{H_t} [=0] \]

\[ \mathbf{T_t}, \mathbf{V_t} [=S_t S_t'], \mathbf{S_t} [=Cholesky(V)] \]

\[ \mathbf{a_{0}}[=0], \mathbf{P_*} [=0], \mathbf{B} [=0], \mathbf{P_\infty} [=BB'] \]

\section{Other notations}\label{other-notations}

The following notations are used in all the pages related to the ssf
framework:

\[ a_{t+k|t}=E\left(\alpha_{t+k} | y_0 \cdots y_{t}\right)\]

\[ P_{t+k|t}=cov\left(\alpha_{t+k} | y_0 \cdots y_{t}\right)\]

\[ a_{t+1}=a_{t+1|t}, \: P_{t+1}=P_{t+1|t} \]

Finally, it should be noted that all the indexes of arrays or matrices
start at 0 (Java and C++ convention).

\chapter{Functional forms}\label{functional-forms}

\section{Functional forms}\label{functional-forms-1}

State space forms and the related algorithms focus on the state vectors
and their (conditional) distribution, and on the relationships between
those vectors at different time points. For instance, using obvious
notations, we will consider:

\begin{longtable}[]{@{}
  >{\raggedright\arraybackslash}p{(\columnwidth - 2\tabcolsep) * \real{0.2571}}
  >{\raggedright\arraybackslash}p{(\columnwidth - 2\tabcolsep) * \real{0.7429}}@{}}
\toprule\noalign{}
\begin{minipage}[b]{\linewidth}\raggedright
Operations
\end{minipage} & \begin{minipage}[b]{\linewidth}\raggedright
Formulae
\end{minipage} \\
\midrule\noalign{}
\endhead
\bottomrule\noalign{}
\endlastfoot
Initialization & \(a_0 = a_{0\vert -1}\) \\
Prediction & \(a_{t \vert t}\rightarrow a_{t+1 \vert t}=a_{t+1}\) \\
Update & \(a_{t \vert t-1}=a_t \rightarrow a_{t \vert t}\) \\
Prediction error & \(e_t=y_t- \hat y_{t \vert t-1}\) \\
Smoothing & \(a_{t \vert n} \rightarrow a_{t-1\vert n}\) \\
\ldots{} & \ldots{} \\
\end{longtable}

The relationships considered above are usually expressed under the form
of matrices, which are often sparse. All things considered, matrices are
only a convenient way for describing linear transformations. By
replacing matrix computations (at least the most frequent and/or the
most expansive ones) with equivalent functions, it is possible to
achieve substantial performance gain.

This is the basic principle of the JD+ state space framework: every type
of model will have to provide a set of functions that allows an
efficient computation of the different algorithms considered in the
framework.

We shall clarify that point by an example. The prediction step is
defined by the equations:

\[ a_{t+1 \vert t}=T_t a_{t \vert t}, \quad P_{t+1 \vert t}=T_t P_{t \vert t} T_t'+V_t=T_t \left(T_t P_{t \vert t} \right)'+V_t \]

The transformation \(f_{T_t} (x)\) of an array (and by extension, of a
matrix) by means of the matrix \(T_t\) plays thus a central role in the
forecasting step:

\begin{itemize}
\tightlist
\item
  Apply \(f_{T_t}\) on \(a\)
\item
  Apply \(f_{T_t}\) on each column of \(P\) ; we get \(Q\)
\item
  Apply \(f_{T_t}\) on each row of \(Q\) ; we get \(O\)
\item
  Add \(V_t\) to \(O\)
\end{itemize}

In the case of (partially) time invariant systems, we will usually omit
in the documentation of the framework the subscript \(_t\) of the
different arrays/matrices/functions.

Using matrix computation, the transformation \(f_{T_t}\) needs
\(r \times r\) multiplications, where \(r\) is the length of the state
array. In many cases, the functional form will involve at most \(r\)
multiplications and much less memory traffic.

For instance, in some SSF forms of ARIMA models (see Gomez-Maravall,
1994), the function \(y=f_T(x)\) is defined by

\[ y_i = \begin{cases} x_{i+1} & 0 \le i < r-1  \\ -\sum_{k=1}^p {\varphi_k x_{r-k} } & i =r-1\end{cases} \]

which corresponds to the transformation matrix

\[
\begin{pmatrix}
0 & 1 & 0 & \cdots & 0 \\
0 & 0 & 1 & \cdots & 0 \\
\vdots & \vdots & \ddots & \ddots & 0 \\
0 & \cdots & \cdots & 0 & 1 \\
0 & \cdots & -\phi_p & - \cdots & -\phi_1
\end{pmatrix}
\]

It involves only \(p\) (order of the autoregressive polynomial)
multiplications.

As soon as direct matrix computations are avoided, the matrices of the
system themselves don't usually need to be created. That leads to
another substantial gain in time and in memory space/traffic, especially
in the case of time dependent systems.

\chapter{SSF structure}\label{ssf-structure}

\chapter{Java design}\label{java-design}

test




\end{document}
