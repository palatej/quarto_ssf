% Options for packages loaded elsewhere
\PassOptionsToPackage{unicode}{hyperref}
\PassOptionsToPackage{hyphens}{url}
\PassOptionsToPackage{dvipsnames,svgnames,x11names}{xcolor}
%
\documentclass[
  letterpaper,
  DIV=11,
  numbers=noendperiod]{scrreprt}

\usepackage{amsmath,amssymb}
\usepackage{iftex}
\ifPDFTeX
  \usepackage[T1]{fontenc}
  \usepackage[utf8]{inputenc}
  \usepackage{textcomp} % provide euro and other symbols
\else % if luatex or xetex
  \usepackage{unicode-math}
  \defaultfontfeatures{Scale=MatchLowercase}
  \defaultfontfeatures[\rmfamily]{Ligatures=TeX,Scale=1}
\fi
\usepackage{lmodern}
\ifPDFTeX\else  
    % xetex/luatex font selection
\fi
% Use upquote if available, for straight quotes in verbatim environments
\IfFileExists{upquote.sty}{\usepackage{upquote}}{}
\IfFileExists{microtype.sty}{% use microtype if available
  \usepackage[]{microtype}
  \UseMicrotypeSet[protrusion]{basicmath} % disable protrusion for tt fonts
}{}
\makeatletter
\@ifundefined{KOMAClassName}{% if non-KOMA class
  \IfFileExists{parskip.sty}{%
    \usepackage{parskip}
  }{% else
    \setlength{\parindent}{0pt}
    \setlength{\parskip}{6pt plus 2pt minus 1pt}}
}{% if KOMA class
  \KOMAoptions{parskip=half}}
\makeatother
\usepackage{xcolor}
\setlength{\emergencystretch}{3em} % prevent overfull lines
\setcounter{secnumdepth}{5}
% Make \paragraph and \subparagraph free-standing
\makeatletter
\ifx\paragraph\undefined\else
  \let\oldparagraph\paragraph
  \renewcommand{\paragraph}{
    \@ifstar
      \xxxParagraphStar
      \xxxParagraphNoStar
  }
  \newcommand{\xxxParagraphStar}[1]{\oldparagraph*{#1}\mbox{}}
  \newcommand{\xxxParagraphNoStar}[1]{\oldparagraph{#1}\mbox{}}
\fi
\ifx\subparagraph\undefined\else
  \let\oldsubparagraph\subparagraph
  \renewcommand{\subparagraph}{
    \@ifstar
      \xxxSubParagraphStar
      \xxxSubParagraphNoStar
  }
  \newcommand{\xxxSubParagraphStar}[1]{\oldsubparagraph*{#1}\mbox{}}
  \newcommand{\xxxSubParagraphNoStar}[1]{\oldsubparagraph{#1}\mbox{}}
\fi
\makeatother


\providecommand{\tightlist}{%
  \setlength{\itemsep}{0pt}\setlength{\parskip}{0pt}}\usepackage{longtable,booktabs,array}
\usepackage{calc} % for calculating minipage widths
% Correct order of tables after \paragraph or \subparagraph
\usepackage{etoolbox}
\makeatletter
\patchcmd\longtable{\par}{\if@noskipsec\mbox{}\fi\par}{}{}
\makeatother
% Allow footnotes in longtable head/foot
\IfFileExists{footnotehyper.sty}{\usepackage{footnotehyper}}{\usepackage{footnote}}
\makesavenoteenv{longtable}
\usepackage{graphicx}
\makeatletter
\def\maxwidth{\ifdim\Gin@nat@width>\linewidth\linewidth\else\Gin@nat@width\fi}
\def\maxheight{\ifdim\Gin@nat@height>\textheight\textheight\else\Gin@nat@height\fi}
\makeatother
% Scale images if necessary, so that they will not overflow the page
% margins by default, and it is still possible to overwrite the defaults
% using explicit options in \includegraphics[width, height, ...]{}
\setkeys{Gin}{width=\maxwidth,height=\maxheight,keepaspectratio}
% Set default figure placement to htbp
\makeatletter
\def\fps@figure{htbp}
\makeatother
% definitions for citeproc citations
\NewDocumentCommand\citeproctext{}{}
\NewDocumentCommand\citeproc{mm}{%
  \begingroup\def\citeproctext{#2}\cite{#1}\endgroup}
\makeatletter
 % allow citations to break across lines
 \let\@cite@ofmt\@firstofone
 % avoid brackets around text for \cite:
 \def\@biblabel#1{}
 \def\@cite#1#2{{#1\if@tempswa , #2\fi}}
\makeatother
\newlength{\cslhangindent}
\setlength{\cslhangindent}{1.5em}
\newlength{\csllabelwidth}
\setlength{\csllabelwidth}{3em}
\newenvironment{CSLReferences}[2] % #1 hanging-indent, #2 entry-spacing
 {\begin{list}{}{%
  \setlength{\itemindent}{0pt}
  \setlength{\leftmargin}{0pt}
  \setlength{\parsep}{0pt}
  % turn on hanging indent if param 1 is 1
  \ifodd #1
   \setlength{\leftmargin}{\cslhangindent}
   \setlength{\itemindent}{-1\cslhangindent}
  \fi
  % set entry spacing
  \setlength{\itemsep}{#2\baselineskip}}}
 {\end{list}}
\usepackage{calc}
\newcommand{\CSLBlock}[1]{\hfill\break\parbox[t]{\linewidth}{\strut\ignorespaces#1\strut}}
\newcommand{\CSLLeftMargin}[1]{\parbox[t]{\csllabelwidth}{\strut#1\strut}}
\newcommand{\CSLRightInline}[1]{\parbox[t]{\linewidth - \csllabelwidth}{\strut#1\strut}}
\newcommand{\CSLIndent}[1]{\hspace{\cslhangindent}#1}

\KOMAoption{captions}{tableheading}
\makeatletter
\@ifpackageloaded{bookmark}{}{\usepackage{bookmark}}
\makeatother
\makeatletter
\@ifpackageloaded{caption}{}{\usepackage{caption}}
\AtBeginDocument{%
\ifdefined\contentsname
  \renewcommand*\contentsname{Table of contents}
\else
  \newcommand\contentsname{Table of contents}
\fi
\ifdefined\listfigurename
  \renewcommand*\listfigurename{List of Figures}
\else
  \newcommand\listfigurename{List of Figures}
\fi
\ifdefined\listtablename
  \renewcommand*\listtablename{List of Tables}
\else
  \newcommand\listtablename{List of Tables}
\fi
\ifdefined\figurename
  \renewcommand*\figurename{Figure}
\else
  \newcommand\figurename{Figure}
\fi
\ifdefined\tablename
  \renewcommand*\tablename{Table}
\else
  \newcommand\tablename{Table}
\fi
}
\@ifpackageloaded{float}{}{\usepackage{float}}
\floatstyle{ruled}
\@ifundefined{c@chapter}{\newfloat{codelisting}{h}{lop}}{\newfloat{codelisting}{h}{lop}[chapter]}
\floatname{codelisting}{Listing}
\newcommand*\listoflistings{\listof{codelisting}{List of Listings}}
\makeatother
\makeatletter
\makeatother
\makeatletter
\@ifpackageloaded{caption}{}{\usepackage{caption}}
\@ifpackageloaded{subcaption}{}{\usepackage{subcaption}}
\makeatother

\ifLuaTeX
  \usepackage{selnolig}  % disable illegal ligatures
\fi
\usepackage{bookmark}

\IfFileExists{xurl.sty}{\usepackage{xurl}}{} % add URL line breaks if available
\urlstyle{same} % disable monospaced font for URLs
\hypersetup{
  pdftitle={Quarto-SSF},
  pdfauthor={Jean Palate},
  colorlinks=true,
  linkcolor={blue},
  filecolor={Maroon},
  citecolor={Blue},
  urlcolor={Blue},
  pdfcreator={LaTeX via pandoc}}


\title{Quarto-SSF}
\author{Jean Palate}
\date{2026-02-01}

\begin{document}
\maketitle

\renewcommand*\contentsname{Table of contents}
{
\hypersetup{linkcolor=}
\setcounter{tocdepth}{2}
\tableofcontents
}

\bookmarksetup{startatroot}

\chapter*{Index}\label{index}
\addcontentsline{toc}{chapter}{Index}

\markboth{Index}{Index}

\bookmarksetup{startatroot}

\chapter{}\label{section}

\bookmarksetup{startatroot}

\chapter*{References}\label{references}
\addcontentsline{toc}{chapter}{References}

\markboth{References}{References}

\phantomsection\label{refs}
\begin{CSLReferences}{0}{1}
\end{CSLReferences}

\part{Introduction}

\chapter{Overview}\label{overview}

\section{Description of the general
model}\label{description-of-the-general-model}

The \texttt{general\ linear\ gaussian} state-space model can be written
in many different ways. The form considered in JD+ (\(\ge 3.0\)) is
presented below:

\[ y_t = Z_t \alpha_t + \epsilon_t,\quad \epsilon_t \sim N\left(0, \sigma^2 H_t\right),\quad t> 0\]

\[ \alpha_{t+1} = T_t \alpha_t + \mu_t, \quad \mu_t \sim N \left(0, \sigma^2 V_t \right),\quad t \ge 0 \]

\(y_t\) is the observation at period t, \(\alpha_t\) is the state
vector. \(\epsilon_t, \mu_t\) are assumed to be serially independent at
all leads and lags and independent from each other.\\
In the case of multi-variate models, \(y_t\) is a vector of
observations. However, in most cases, we will use the univariate
approach by considering the observations one by one (univariate handling
of multi-variate models).

The innovations of the state equation will be modeled as

\[ \mu_t = S_t \xi_t, \quad \xi_t \sim N\left( 0, \sigma^2 I\right) \]

In other words, \(V_t=S_t S_t'\)

The initial (\(\equiv t=0\)) conditions of the filter are defined as
follows:

\[ \alpha_{0} = b + B\delta + \mu_{0}\]

\(\delta\) is unknown or
\(\delta \sim N\left(0, \kappa \sigma^2 I \right)\), where \(\kappa\) is
arbitrary large. \(\mu_{0} \sim N\left(0, \sigma^2 P_*\right)\). So,
\(P_*\) is the variance of the stationary part of the initial state
vector and \(B\), which must be a full rank matrix, models the diffuse
part. We write \(BB'=P_\infty\). It should be noted that applying an
orthogonal transformation to \(B\) or multiplying it by a constant will
give exactly the same initialization. Moreover, when the initial state
is completely diffuse, \(\mu_0\) is not relevant and it can be safely
considered as \(0\). Finally, it should be noted that imposing that the
covariance of \(\delta\) is proportional to \(I\) is not an actual
restriction: in the hypothesis of a more general non singular matrix, we
just should pre-multiply \(\delta\) by the inverse of the Cholesky
factor of that matrix and post-multiply \(B\) by that factor to get an
equivalent initialization.

The definition used in JD+ is quasi-identical to that of Durbin and
Koopman{[}1{]}.

In summary, apart from the scaling factor \(\sigma^2\) which can be
given or estimated, the model is completely defined by the following
quantities (possible default values are indicated in brackets):

\[ \mathbf{Z_t}, \mathbf{H_t} [=0] \]

\[ \mathbf{T_t}, \mathbf{V_t} [=S_t S_t'], \mathbf{S_t} [=Cholesky(V)] \]

\[ \mathbf{b}[=0], \mathbf{P_*} [=0], \mathbf{B} [=0], \mathbf{P_\infty} [=BB'] \]

It should be noted that all the indexes of arrays or matrices start at
\(0\) (Java and C++ convention). The following notations are also used
in all the pages related to the ssf framework:

\[ a_{t+k|t}=E\left(\alpha_{t+k} | y_0 \cdots y_{t}\right)\]

\[ P_{t+k|t}=cov\left(\alpha_{t+k} | y_0 \cdots y_{t}\right)\]

\[ a_{t+1}=a_{t+1|t}, \: P_{t+1}=P_{t+1|t} \] \[ 
\begin{align*}
y_t - E(y_t| y_0 \cdots y_t-1) &= e_t \\
var(e_t) &= f_t \\
\end{align*}
\] or, in the multivariate case

\[ 
var(e_t) = F_t = R_t R_t'
\]

\section{Algorithms}\label{algorithms}

State space models are convenient representations of time series to
derive multiple properties. Algorithms has been developed around them
for likelihood evaluation (and so, for parameters estimation), for
filtering/forecasting, for smoothing/missing values estimation, for
simulation\ldots{}

Multiple variations of those algorithms have been developed, to favor
either speed, memory use, robustness or accuracy.

The JD+ framework provides several family of those algorithms, mainly
based on approaches developed by Durbin-Koopman (DK), De Jong (AKF or
Augmented Kalman Filter), Kailath (Array filters) or Chandrasekhar
filters (CKMS or Chandrasekhar-Kailath-Morf-Sidhu). They will be
described in the chapter 3.

\section{Design}\label{design}

\chapter{Design}\label{design-1}

\section{Functional forms}\label{functional-forms}

State space forms and the related algorithms focus on the state vectors
and their (conditional) distribution, and on the relationships between
those vectors at different time points. For instance, using obvious
notations, we will consider:

\begin{longtable}[]{@{}
  >{\raggedright\arraybackslash}p{(\columnwidth - 2\tabcolsep) * \real{0.2571}}
  >{\raggedright\arraybackslash}p{(\columnwidth - 2\tabcolsep) * \real{0.7429}}@{}}
\toprule\noalign{}
\begin{minipage}[b]{\linewidth}\raggedright
Operations
\end{minipage} & \begin{minipage}[b]{\linewidth}\raggedright
Formulae
\end{minipage} \\
\midrule\noalign{}
\endhead
\bottomrule\noalign{}
\endlastfoot
Initialization & \(a_0 = a_{0\vert -1}\) \\
Prediction & \(a_{t \vert t}\rightarrow a_{t+1 \vert t}=a_{t+1}\) \\
Update & \(a_{t \vert t-1}=a_t \rightarrow a_{t \vert t}\) \\
Prediction error & \(e_t=y_t- \hat y_{t \vert t-1}\) \\
Smoothing & \(a_{t \vert n} \rightarrow a_{t-1\vert n}\) \\
\ldots{} & \ldots{} \\
\end{longtable}

The relationships considered above are usually expressed under the form
of matrices, which are often sparse. All things considered, matrices are
only a convenient way for describing linear transformations. By
replacing matrix computations (at least the most frequent and/or the
most expansive ones) with equivalent functions, it is possible to
achieve substantial performance gain.

This is the basic principle of the JD+ state space framework: every type
of model will have to provide a set of functions that allows an
efficient computation of the different algorithms considered in the
framework.

We shall clarify that point by an example. The prediction step is
defined by the equations:

\[ a_{t+1 \vert t}=T_t a_{t \vert t}, \quad P_{t+1 \vert t}=T_t P_{t \vert t} T_t'+V_t=T_t \left(T_t P_{t \vert t} \right)'+V_t \]

The transformation \(f_{T_t} (x)\) of an array (and by extension, of a
matrix) by means of the matrix \(T_t\) plays thus a central role in the
forecasting step:

\begin{itemize}
\tightlist
\item
  Apply \(f_{T_t}\) on \(a\)
\item
  Apply \(f_{T_t}\) on each column of \(P\) ; we get \(Q\)
\item
  Apply \(f_{T_t}\) on each row of \(Q\) ; we get \(O\)
\item
  Add \(V_t\) to \(O\)
\end{itemize}

In the case of (partially) time invariant systems, we will usually omit
in the documentation of the framework the subscript \(_t\) of the
different arrays/matrices/functions.

Using matrix computation, the transformation \(f_{T_t}\) needs
\(r \times r\) multiplications, where \(r\) is the length of the state
array. In many cases, the functional form will involve at most \(r\)
multiplications and much less memory traffic.

For instance, in some SSF forms of ARIMA models (see Gomez-Maravall,
1994), the function \(y=f_T(x)\) is defined by

\[ y_i = \begin{cases} x_{i+1} & 0 \le i < r-1  \\ -\sum_{k=1}^p {\varphi_k x_{r-k} } & i =r-1\end{cases} \]

which corresponds to the transformation matrix

\[
\begin{pmatrix}
0 & 1 & 0 & \cdots & 0 \\
0 & 0 & 1 & \cdots & 0 \\
\vdots & \vdots & \ddots & \ddots & 0 \\
0 & \cdots & \cdots & 0 & 1 \\
0 & \cdots & -\phi_p & - \cdots & -\phi_1
\end{pmatrix}
\]

It involves only \(p\) (order of the autoregressive polynomial)
multiplications.

As soon as direct matrix computations are avoided, the matrices of the
system themselves don't usually need to be created. That leads to
another substantial gain in time and in memory space/traffic, especially
in the case of time dependent systems.

\chapter{SSF structure}\label{ssf-structure}

\chapter{Java design}\label{java-design}

\part{SSF blocks}

\chapter{Atomic blocks}\label{atomic-blocks}

\section{ARMA model}\label{arma-model}

\subsection{Introduction}\label{introduction-1}

The ARMA process is defined by

\[\Phi\left(B\right)y_t=\Theta\left(B\right)\epsilon_t \]

where:

\[\Phi\left(B\right)=1+\varphi_1 B + \cdots + \varphi_p B^p \]

\[\Theta\left(B\right)=1+\theta_1 B + \cdots + \theta_q B^q \]

are the auto-regressive and the moving average polynomials.

The MA representation of the model is
\(y_t=\sum_{i=0}^\infty {\psi_i \epsilon_{t-i}}\). Let \(\gamma_i\) be
the autocovariances of the model. We also define:
\(r=\max\left(p, q+1\right), \quad s=r-1\).

The state-space model can be written in different ways. JD+ provides two
of them

\subsection{Representation I}\label{representation-i}

\subsubsection{State vector}\label{state-vector}

\[ \alpha_t= \begin{pmatrix} y_t \\ y_{t+1|t} \\ \vdots \\ y_{t+s|t} \end{pmatrix}\]

where \(y_{t+i|t}\) is the orthogonal projection of \(y_{t+i}\) on the
subspace generated by \({y\left(s\right):s \leq t}\).Thus, it is the
forecast function with respect to the semi-infinite sample. We also have
that \(y_{t+i|t} = \sum_{j=i}^\infty {\psi_j \epsilon_{t+i-j}}\)

\subsubsection{Dynamics}\label{dynamics}

\[ T_t = \begin{pmatrix}0 &1 & 0 & \cdots & 0  \\0& 0 & 1 & \cdots & 0\\ \vdots & \vdots & \vdots & \ddots & \vdots\\ 0 & 0 & 0 & \cdots & 1\\
-\varphi_r & \cdots  & \cdots & \cdots &-\varphi_1 \end{pmatrix}\]

with \(\varphi_{j}=0\) for \(j>p\)

\[ S_t = \begin{pmatrix}1 \\ \psi_1 \\ \vdots\\ \psi_s \end{pmatrix} \]

\[ V_t = S S' \]

\subsubsection{Default measurement}\label{default-measurement}

\[ Z_t = \begin{pmatrix} 1 & 0 & \cdots & 0\end{pmatrix}\]

\[ h_t = 0 \]

\subsubsection{Initialization}\label{initialization}

\[ \alpha_{-1} = \begin{pmatrix}0 \\ 0 \\ \vdots\\ 0 \end{pmatrix} \]

\[ P_{*} = \Omega \]

\(\Omega\) is the unconditional covariance of the state array; it can be
easily derived using the MA representation. We have:

\[ \Omega\left(i,0\right) = \gamma_i \]

\[ \Omega\left(i,j\right) = \Omega\left(i-1,j-1\right)-\psi_i \psi_j \]

\subsection{Representation II}\label{representation-ii}

We provide below an alternative state space representation of ARMA
models

\subsubsection{State vector:}\label{state-vector-1}

\[ \alpha_t= \begin{pmatrix} y_t \\ 
-\varphi_2 y_{t-1}-\dots-\varphi_r y_{t-r+1}+\theta_1 \epsilon_t+ \dots + \theta_{r-1} \epsilon_{t-r+2}\\ -\varphi_3 y_{t-1}-\dots-\varphi_r y_{t-r+2}+\theta_2 \epsilon_t+ \dots + \theta_{r-1} \epsilon_{t-r+3}\\ \vdots \\ 
-\varphi_r y_{t-1}+\theta_{r-1} \epsilon_{t}
\end{pmatrix}\]

\subsubsection{Dynamics}\label{dynamics-1}

\[ T_t = \begin{pmatrix}-\varphi_1 &1 & 0 & \cdots & 0  \\
-\varphi_2& 0 & 1 & \cdots & 0\\ 
\vdots & \vdots & \vdots & \ddots & \vdots\\ 
\vdots & 0 & 0 & \cdots & 1\\
-\varphi_r & 0  & \cdots & \cdots &0 \end{pmatrix}\]

\[ S_t = \begin{pmatrix}1 \\ \theta_1 \\ \vdots\\ \theta_r \end{pmatrix} \]

\[ V_t = S S' \]

\subsubsection{Default measurement}\label{default-measurement-1}

\[ Z_t = \begin{pmatrix} 1 & 0 & \cdots & 0\end{pmatrix}\]

\[ h_t = 0 \]

\subsubsection{Initialization}\label{initialization-1}

\[ \alpha_{-1} = \begin{pmatrix}0 \\ 0 \\ \vdots\\ 0 \end{pmatrix} \]

\[ P_{*} = \Omega \]

\(\Omega\) is the unconditional covariance of the state array; it can be
easily derived using the MA representation. We have:

\[ \Omega\left(i,0\right) = \gamma_i \]

\[ \Omega\left(i,j\right) = \Omega\left(i-1,j-1\right)-\psi_i \psi_j \]
\#\#\# Implementation

\section{Time dependent ARMA model}\label{time-dependent-arma-model}

\subsection{Introduction}\label{introduction-2}

The time dependent ARMA process (TDARMA) is defined by

\[ \Phi_t\left(B\right)y_t=\Theta_t\left(B\right)\epsilon_t \]

where:

\[\Phi_t\left(B\right)=1+\varphi_{t,1} B + \cdots + \varphi_{t,p} B^p \]

\[\Theta_t\left(B\right)=1+\theta_{t,1} B + \cdots + \theta_{t,q} B^q \]

are the time dependent auto-regressive and the moving average
polynomials.

\subsection{State vector:}\label{state-vector-2}

\[ \alpha_t= \begin{pmatrix} y_t \\ 
-\varphi_{t+1,2} y_{t-1}-\dots-\varphi_{t+1,r} y_{t-r+1}+\theta_{t+1,1} \epsilon_t+ \dots + \theta_{t+1, r-1} \epsilon_{t-r+2}\\ -\varphi_{t+2,3} y_{t-1}-\dots-\varphi_{t+2,r} y_{t-r+2}+\theta_{t+2,2 }\epsilon_t+ \dots + \theta_{t+2,r-1} \epsilon_{t-r+3}\\ \vdots \\ 
-\varphi_{t+r-1,r} y_{t-1}+\theta_{t+r-1,r-1} \epsilon_{t}
\end{pmatrix}\]

\subsection{Dynamics}\label{dynamics-2}

\[ \alpha_{t+1} = T_t \alpha_t + \begin{pmatrix} 1 \\ \theta_{t+2,1} \\ \theta_{t+3,2} \\ \vdots \\  \theta_{t+r, r-1} \\ \end{pmatrix} \epsilon_{t+1} \]

\[ T_t = \begin{pmatrix}-\varphi_{t+1,1} &1 & 0 & \cdots & 0  \\
-\varphi_{t+2,2} & 0 & 1 & \cdots & 0\\ 
\vdots & \vdots & \vdots & \ddots & \vdots\\ 
\vdots & 0 & 0 & \cdots & 1\\
-\varphi_{t+r,r} & 0  & \cdots & \cdots &0 \end{pmatrix}\]

\[ S_t = \begin{pmatrix}\sigma_{t+1}\\ \sigma_{t+1} \theta_{t+2,1} \\ \vdots\\ \sigma_{t+1} \theta_{t+r,r-1} \end{pmatrix} \]

\[ V_t = S S' \]

\subsection{Default measurement}\label{default-measurement-2}

\[ Z_t = \begin{pmatrix} 1 & 0 & \cdots & 0\end{pmatrix}\]

\[ h_t = 0 \]

\subsection{Initialization}\label{initialization-2}

\[ \alpha_0= \begin{pmatrix} y_0 \\ 
-\varphi_{1,2} y_{-1}-\dots-\varphi_{1,r} y_{-r+1}+\theta_{1,1} \epsilon_0+ \dots + \theta_{1, r-1} \epsilon_{-r+2}\\ -\varphi_{2,3} y_{-1}-\dots-\varphi_{2,r} y_{-r+2}+\theta_{2,2 }\epsilon_0+ \dots + \theta_{2,r-1} \epsilon_{-r+3}\\ \vdots \\ 
-\varphi_{r-1,r} y_{-1}+\theta_{r-1,r-1} \epsilon_{0}
\end{pmatrix}\]

The covariance of \(\alpha_0\) can be easily computed, taking into
account that \(y_0, \dots,y_{r-1}\) and
\(\epsilon_0, \dots, \epsilon_{r-1}\) are generated by the same
processes and that their variances/covariances are:

\[ cov(y_i, y_j)=\gamma(i-j)\]

\[ cov(y_i, \epsilon_j) = \psi_{j-i}\sigma^2, j\le i\]

\[ cov(\epsilon_i, \epsilon_j)=\sigma^2, i=j \]

\section{ARIMA model}\label{arima-model}

\subsection{Introduction}\label{introduction-3}

We consider that the auto-regressive polynomial contains some unit
roots. It can be factorized in a stationary polynomial (defined by the
roots outside the unit circle) and in a non-stationary polynomial
(defined by the roots on the unit circle), which is notated:

\[ \Delta\left(B\right) = 1 + \delta_1 B + \cdots + \delta_d B^d \]

The state space form of an ARIMA model is similar to the state space
form of an ARMA model, except for its initialization.

\subsection{Initialization}\label{initialization-3}

The initial conditions can be written as follows:

\[ \alpha_{-1} = \begin{pmatrix}1 \\ 0 \\ \vdots\\ 0 \end{pmatrix} \]

\[ P_{*} = \Sigma \Omega \Sigma' \]

\[ B = \Lambda \]

\[ P_{\infty}= \Lambda \Lambda' \]

\(\Omega\) is the unconditional covariance of the state array of the
stationary model.

\[ \Sigma = \begin{pmatrix} 1 & 0 & \cdots & 0 \\ \lambda_1 & 1 & \cdots & 0 \\ \lambda_2 & \lambda_1 & \cdots & \vdots\\ \vdots & \vdots & \vdots & \vdots \\ \lambda_{r-1} & \lambda_{r-2} & \cdots & \lambda_{1} \end{pmatrix} \]

where \(\lambda_{i}\) are generated by
\(\frac{1}{\Delta\left(B\right)}\)

\(\Lambda\) is a r x d matrix; its first d rows form an identity matrix;
other cells are defined by the recursive relationship:

\[\Lambda \left(i,j\right) = -\sum_{k=1}^d {\delta_k \Lambda \left(i-k,j\right)}\]

\section{Noise}\label{noise}

\subsection{State vector}\label{state-vector-3}

\subsection{Dynamics}\label{dynamics-3}

\subsection{Default measurement}\label{default-measurement-3}

\subsection{Initialization}\label{initialization-4}

\subsection{Implementation}\label{implementation}

\section{Local level}\label{local-level}

\subsubsection{Description}\label{description}

The local level block describes a random walk component.

\[
\begin{align*}
l_{t+1} &= l_t + \epsilon_t \\
\epsilon_t &\sim N(0, \sigma^2_l)
\end{align*}
\]

\subsection{State vector}\label{state-vector-4}

The state block is \(\alpha_t=\begin{pmatrix} l_t  \end{pmatrix}\)

\subsection{Dynamics}\label{dynamics-4}

\[
\begin{align*}
T_t &= \begin{pmatrix} 1 \end{pmatrix} \\
V_t &= \begin{pmatrix}\sigma^2_l \end{pmatrix}\\
S_t &= \begin{pmatrix}\sigma_l \end{pmatrix}
\end{align*}
\]

\subsection{Default measurement}\label{default-measurement-4}

\[
\begin{align*}
Z_t &= \begin{pmatrix} 1 \end{pmatrix} \end{align*} 
\]

\subsection{Diffuse initialization}\label{diffuse-initialization}

\[
\begin{align*}
a_0 &= \begin{pmatrix} 0 \end{pmatrix} \\
P_* &= \begin{pmatrix} 0 \end{pmatrix}\\
B &= \begin{pmatrix} 1 \end{pmatrix} \\
P_\infty &= \begin{pmatrix} 1 \end{pmatrix}
\end{align*}
\]

\subsection{Fixed initialization}\label{fixed-initialization}

\[
\begin{align*}
a_0 &= \begin{pmatrix} l_0 \end{pmatrix} \\
P_* &= \begin{pmatrix} \sigma_{l} \end{pmatrix}\\
B &= \begin{pmatrix} 0 \end{pmatrix} \\
P_\infty &= \begin{pmatrix} 0 \end{pmatrix}
\end{align*}
\]

\subsection{Implementation}\label{implementation-1}

See \texttt{jdplus.toolkit.base.core.ssf.sts.LocalLevel}

\section{Local linear trend}\label{local-linear-trend}

\subsection{Description}\label{description-1}

The local linear trend block describes the following trend component:

\[
\begin{align*}
l_{t+1} &= l_t + n_t +  \epsilon_t \\
n_{t+1} &= n_t + \mu_t \\
\epsilon_t &\sim N(0, \sigma^2_l) \\
\mu_t &\sim N(0, \sigma^2_n)
\end{align*}
\]

\subsection{State vector}\label{state-vector-5}

The state block is

\[\alpha_t=
\begin{pmatrix} 
l_t \\ 
n_t  
\end{pmatrix}
\] \#\#\# Dynamics

\[
\begin{align*}
T_t &= \begin{pmatrix} 1 && 1 \\ 0 && 1 \end{pmatrix} \\
V_t &= \begin{pmatrix} \sigma^2_l && 0 \\ 0 && \sigma^2_n \end{pmatrix} \\
S_t &= \begin{pmatrix} \sigma_l && 0 \\ 0 && \sigma_n \end{pmatrix} 
\end{align*}
\]

\subsection{Default measurement}\label{default-measurement-5}

\[
\begin{align*}
Z_t &= \begin{pmatrix} 1 & 0 \end{pmatrix} 
\end{align*} 
\]

\subsection{Initialization}\label{initialization-5}

\[
\begin{align*}
a_0 &= \begin{pmatrix} 0 \\ 0  \end{pmatrix} \\
P_* &=  \begin{pmatrix} 0 && 0 \\ 0 && 0 \end{pmatrix}\\
B &= \begin{pmatrix} 1 && 0 \\ 0 && 1 \end{pmatrix} \\
P_{\infty} &= \begin{pmatrix} 1 && 0 \\ 0 && 1 \end{pmatrix}
\end{align*}
\]

\subsection{Parameters}\label{parameters}

\[ \sigma^2_l \ge 0,\quad \sigma^2_n \ge 0 \]

\subsection{Implementation}\label{implementation-2}

See \texttt{jdplus.toolkit.base.core.ssf.sts.LocalLinearTrend}

\section{Seasonal component}\label{seasonal-component}

\subsection{Introduction}\label{introduction-4}

The seasonal component provided in JD+ is based on Proietti{[}2000{]}

We consider that it is composed of latent variables corresponding to
each period, which follow a multivariate random walk. We also impose
that the sum of the latent variables is 0 at each point time.

\subsection{State vector}\label{state-vector-6}

For a periodicity of \(s\), we define the state space form of the
seasonal component as follows (West-Harrison form):

We consider first the state vector \(\tilde \gamma_t\) such that
\(\tilde \gamma_{it}\) is the (unobserved) seasonal component of season
\(i\) observed at time \(t\).

Using that definition would imply a time varying measurement error. To
avoid it, we use the state vector \(\gamma_t\) which is a rotation of
\(\tilde \gamma_t\) such that the first component corresponds to the
current period. More precisely,
\(\gamma_{it} =\tilde\gamma_{\left( \left(t-i\right) \mod s \right), t}\).

Moreover, the constraint that the sum of the latent variables is 0
allows us to consider only \(s-1\) periods (the missing one being the
opposite of the sum of all the other periods).

\subsection{Dynamics}\label{dynamics-5}

\[ T_t = \begin{pmatrix} 0 & 1 & \cdots & 0 \\ \vdots & \ddots & \ddots & \vdots \\ 0 &  \cdots & 0 & 1  \\ -1 & -1 & \cdots & -1\end{pmatrix}\]

The covariance matrix of the innovations can take various form. Using
the terminology of Proietti, we consider:

\begin{longtable}[]{@{}
  >{\raggedright\arraybackslash}p{(\columnwidth - 4\tabcolsep) * \real{0.3333}}
  >{\raggedright\arraybackslash}p{(\columnwidth - 4\tabcolsep) * \real{0.3333}}
  >{\raggedright\arraybackslash}p{(\columnwidth - 4\tabcolsep) * \real{0.3333}}@{}}
\toprule\noalign{}
\begin{minipage}[b]{\linewidth}\raggedright
Model
\end{minipage} & \begin{minipage}[b]{\linewidth}\raggedright
V
\end{minipage} & \begin{minipage}[b]{\linewidth}\raggedright
S
\end{minipage} \\
\midrule\noalign{}
\endhead
\bottomrule\noalign{}
\endlastfoot
Dummy &
\(\begin{pmatrix} 0 &  \cdots & 0 \\ \vdots & & \vdots\\ 0 &\cdots &\sigma_{seas}^2\end{pmatrix}\)
& \(\begin{pmatrix}0 \\ \vdots \\ \sigma_{seas}\end{pmatrix}\) \\
Crude &
\(\sigma_{seas}^2\begin{pmatrix} 1 &  \cdots & 1 \\ \vdots & & \vdots\\ 1 &\cdots & 1\end{pmatrix}\)
& \(\sigma_{seas}\begin{pmatrix}1 \\ \vdots \\ 1\end{pmatrix}\) \\
HarrisonStevens & \(\sigma\_{seas}\^2\) & \\
Trigonometric & & \\
\end{longtable}

\subsection{Measurement}\label{measurement}

\[ Z_t = \begin{pmatrix} 1 & 0 & \cdots & 0\end{pmatrix}\]

\subsection{Bibliography}\label{bibliography}

Proietti T. (2000), ``Comparing seasonal components for structural time
series models'', International Journal of Forecasting, 16, 2, 247-260.

\section{Cyclical component}\label{cyclical-component}

\subsection{Introduction}\label{introduction-5}

A Cycle component is a special case of an AR2 component, which contains
two complex conjugate roots.

More precisely, it is defined by a period \(\lambda\) and a dumping
factor \(\rho\). If \(\gamma = \frac{2\pi}{\lambda}\), we have that

\[   \begin{pmatrix} y_{t+1} \\ y^*_{t+1}\end {pmatrix} = \begin{pmatrix} \rho \cos{\gamma} &&   \rho \sin{\gamma}\\  -\rho \sin{\gamma}  && \rho \cos{\gamma}\end {pmatrix}\begin{pmatrix} y_t \\ y_t^*\end {pmatrix}+\begin{pmatrix} \epsilon_t \\ \epsilon_t^*\end {pmatrix} \]

Using those notations, the state-space model can be written as follows :

\subsection{State vector:}\label{state-vector-7}

\[ \alpha_t= \begin{pmatrix} y_t \\ y_t^* \end{pmatrix}\]

\subsection{Dynamics}\label{dynamics-6}

\[ T_t = \begin{pmatrix} \rho \cos{\gamma} &&   \rho \sin{\gamma}\\  -\rho \sin{\gamma}  && \rho \cos{\gamma}\end {pmatrix}\]

\[ S_t = \sigma_c \begin{pmatrix} 1 && 0\\  0  && 1\end {pmatrix} \]

\[ V_t = \sigma_c^2\begin{pmatrix} 1 && 0\\  0  && 1\end {pmatrix} \]

\subsection{Measurement}\label{measurement-1}

\[ Z_t = \begin{pmatrix} 1 & 0 \end{pmatrix}\]

\[ h_t = 0 \]

\subsection{Initialization}\label{initialization-6}

\[
\begin{align*}
a_0 &= \begin{pmatrix}0 \\ 0 \end{pmatrix} \\
P_{*} &=  \sigma_c^2\begin{pmatrix} \frac{1}{1-\rho^2} && 0\\  0  && \frac{1}{1-\rho^2}\end {pmatrix}
\end{align*}
\]

\chapter{Derived blocks}\label{derived-blocks}

\part{Algorithms}

\chapter{Ordinary filters}\label{ordinary-filters}

\section{Filters}\label{filters}

\subsection{Univariate model}\label{univariate-model}

\subsubsection{Update step t}\label{update-step-t}

\[
\begin{align*}
e_t &= y_t - Z_t a_t \\
M_t &= P_t Z_t' \\
f_t &= Z_t P_t Z_t' +h_t = M_tZ_t' + h_t \\
a_{t|t} &= a_t + M_t f_t^{-1}e_t \\
P_{t|t}&= P_t - M_t f_t^{-1} M_t'
\end{align*}
\]

\subsubsection{Forecast step t}\label{forecast-step-t}

\[
\begin{align*}
a_{t+1} &= T_t a_{t|t} \\
P_{t+1} &= T_t P_{t|t} T_t' + V_t
\end{align*}
\]

\subsubsection{Compact form}\label{compact-form}

The two steps can be easily combined to give a more compact formulation.

\[
\begin{align*}
e_t &= y_t - Z_t a_t \\
f_t &= Z_t P_t Z_t' + h_t \\
K_t &= T_t P_t Z_t' f_t^{-1} \\
a_{t+1} &= T_t a_t + K_t e_t \\
P_{t+1} &= T_t P_t T_t' - K_t f_t K_t' + V_t
\end{align*}
\]

However, the current implementation uses explicitly the two steps form,
to be able to store either \(a_t\) and \(a_{t|t}\).

\subsection{Multivariate model}\label{multivariate-model}

In the multi-variate case, we use a slightly different (but strictly
equivalent) implementation:

\subsubsection{Update step t}\label{update-step-t-1}

\[
\begin{align*}
e_t &= y_t - Z_t a_t \\
F_t &= Z_t P_t Z_t' + H_t = R_t R_t' \quad(Cholesky) \\
\tilde{K_t} &= P_t Z_t' {R_t'}^{-1} \Leftrightarrow \tilde{K_t} R_t' = P_t Z_t' \\
u_t &= R_t^{-1} e_t \Leftrightarrow R_t u_t = e_t \\
a_{t|t} &= a_t + \tilde{K_t} u_t \\
P_{t|t} &= P_t - \tilde{K_t}\tilde{K_t}'
\end{align*}
\]

\subsubsection{Forecast step t}\label{forecast-step-t-1}

\[
\begin{align*}
a_{t+1} &= T_t a_{t|t} \\
P_{t+1} &= T_t P_{t|t} T_t' + V_t
\end{align*}
\]

\subsubsection{Compact form}\label{compact-form-1}

\[
\begin{align*}
e_t &= y_t - Z_t a_t \\
F_t &= Z_t P_t Z_t' + H_t = R_t R_t' \quad(Cholesky) \\
\tilde{K_t} &=  T_t P_t Z_t' {R_t'}^{-1} \Leftrightarrow \tilde{K_t} R_t' = T_t P_t Z_t' \\
u_t &= R_t^{-1} e_t \Leftrightarrow R_t u_t = e_t \\
a_{t+1} &= T_t a_t + \tilde{K_t} u_t \\
P_{t+1} &= T_t P_{t} T_t' + V_t - \tilde{K_t}\tilde{K_t}' 
\end{align*}
\]

\subsubsection{Special cases}\label{special-cases}

This implementation is robust (the covariance matrices are symmetric by
construction) and makes the computation of the likelihood easy. It also
provides a straightforward solution for singular covariance matrices and
for missing values.

When some observations are missing, we just remove from \(Z_t\) the
corresponding equations.

When the covariance matrix \(F_t\) is singular, some columns of \(R_t\)
are equal to 0. Those columns correspond to ``redundant'' equations. The
corresponding errors \(u_t\) and the corresponding columns in the gain
matrix are undefined (set to 0). The over-determined system
\(R_t u_t = e_t\) imposes some restrictions on the observations, which
can be easily checked (the system must be solvable).

When the measurement errors are diagonal, it should be noted that the
solution based on the Cholesky decomposition is identical to the
so-called univariate treatment of multi-variate models.

The following elemnts are saved in objects of the class
\texttt{MultivariateFilteringResults}

\(a_t, P_t, U_t, R_t, \tilde{K_t}\)

\subsection{Implementation}\label{implementation-3}

The ordinary filter is implemented in the classes
\texttt{jdplus.toolkit.base.core.ssf.univariate.OrdinaryFilter} and
\texttt{jdplus.toolkit.base.core.ssf.multivariate.MultivariateOrdinaryFilter}.

\section{Smoother}\label{smoother}

\subsection{Univariate model}\label{univariate-model-1}

\subsection{Multivriate model}\label{multivriate-model}

\chapter{DK filters}\label{dk-filters}

\chapter{Augmented filter}\label{augmented-filter}

\chapter{Array filters}\label{array-filters}

\chapter{Chandrasekhar filters}\label{chandrasekhar-filters}

\chapter{Likelihood}\label{likelihood}

\section{Stationary models}\label{stationary-models}

When the initial state of the model is completely specified, the Kalman
filter provides the exact likelihood in a trivial way, using the
prediction error decomposition:

\[ p(y_0, \ldots, y_{n-1}) = p(y_0) p(y_1\vert y_0) \ldots p(y_{n-1}\vert y_0 \ldots, y_{n-2})\]

Using the notations defined for the ordinary filter, we have:

\subsection{univariate case:}\label{univariate-case}

\[\log l(\theta, \sigma^2 \vert y)=-1/2\left(n \log(2 \pi) + n \log(\sigma^2)  +  \frac{1}{\sigma^2}\sum_{i=0}^{i<n} e_i^2/f_i + \sum_{i=0}^{i<n} {\log \vert f_i \vert }\right)\]
By concentrating \(\sigma^2\) out of the likelihood and by writing
\(RSS=\sum_{i=1}^n e_i^2/f_i\), we obtain :

\[
\begin{align*}
\hat{\sigma}^2 &= \frac{RSS}{n} \\
\log l_c(\theta \vert y)&=-1/2\left(n \log(2 \pi) + n - n \log(n) + n \log(RSS) + \sum_{i=0}^{i<n} {\log \vert f_i \vert }\right)
\end{align*}
\]

\subsection{multi-variate case:}\label{multi-variate-case}

if we write \(N=\sum_{i=0}^n{T_i}\) where \(T_i\) is the number of
observations for the period \(i\) and \(RSS=\sum_{i=1}^n u_i'u_i\), we
get:

\[
\begin{align*}
\log l(\theta, \sigma^2 \vert y) &=-1/2\left(N \log(2 \pi) + N \log(\sigma^2)  +  \frac{1}{\sigma^2}\sum_{i=0}^{i<n} u_i'u_i + 2\sum_{i=0}^{i<n} {\log \vert R_i \vert }\right) \\
\hat{\sigma}^2 &= \frac{RSS}{N} \\
\log l_c(\theta \vert y) &=-1/2\left(N \log(2 \pi) + N  -N \log(N) + N\log(RSS) + 2\sum_{i=0}^{i<n} {\log \vert R_i \vert }\right) 
\end{align*}
\]

\section{Diffuse models}\label{diffuse-models}

\subsection{State space models as linear regression
models}\label{state-space-models-as-linear-regression-models}

The general linear gaussian state space model can be transformed in the
regression model

\[z = y - c = X \delta + \mu  \] where

\[c_t= Z_t\left[ \left(\prod_{i=0}^{i<t}{T_i}\right)\right]b \quad X_t = Z_t\left[ \left(\prod_{i=0}^{i<t}{T_i}\right)\right]B \]

and the covariance matrix \(\sigma^2\Omega\) of \(\mu\) depends on the
structure of the state space model

\subsection{Profile likelihood}\label{profile-likelihood}

\subsection{Diffuse likelihood}\label{diffuse-likelihood}

\subsection{Marginal likelihood}\label{marginal-likelihood}

\subsection{Implementation}\label{implementation-4}

The univariate version of the transformation of the diffuse elements in
regression variables is implemented by the method
\texttt{jdplus.toolkit.base.core.ssf.diffuseEffects} and the
transformation of \(a_0\) by
\texttt{jdplus.toolkit.base.core.ssf.initialEffect}

When the covariance matrix of the process is defined up to the scaling
factor \(\sigma^2\), we define the diffuse likelihood as

\[l(y\vert \theta, \sigma^2) = \lim_{k \rightarrow \infty} \left(l(y \vert k, \theta,\sigma^2)  (2 \pi \sigma^2 k)^{d/2}\right)\]

This is similar to Francke et al.(2010), (equation 14). Apart from the
scaling factor, it is also identical to the approach of Ansley and Kohn
(1985) (theorem 5.1).

Otherwise, we use the usual definition:

\[l(y\vert \theta) = \lim_{k \rightarrow \infty} \left(l(y \vert k, \theta)  (2 \pi  k)^{d/2}\right)\]

\subsection{Durbin-Koopman approach}\label{durbin-koopman-approach}

The expression of the log-likelihood for stationary model as to be
corrected by the term

\[ -1/2 \sum_{i=1}^d {\log \vert F_{\infty, i} \vert }\]

to get the diffuse likelihood by means of the approach of
\href{dk/md}{Durbin and Koopman}. To simplify the notations, we
considered above that the diffuse initialization is performed with the
first \(d\) observations. See DK for the general case.

\subsection{De Jong approach}\label{de-jong-approach}

\section{Bibliography}\label{bibliography-1}

ANSLEY F. C. and KOHN R. (1985), ``Estimating, filtering and smoothing
in state space models with incompletely specified initial conditions'',
The annals of statistics, vol.~13, n°4, 1286-1316.

DURBIN J. AND KOOPMAN S.J. (2012): ``Time Series Analysis by State Space
Methods'', second edition. Oxford University Press.

FRANCKE, M. K., KOOPMAN, S. J. and DE VOS, A. F. (2010), ``Likelihood
functions for state space models with diffuse initial conditions'',
Journal of Time Series Analysis, 31, 6.

KOHN R. AND ANSLEY F. C. (1985), ``Efficient estimation and prediction
in time series regression models'', Biometrika, 72, 694-697.

\part{Implementations}

\chapter{Java implementation}\label{java-implementation}

\chapter{R implementation}\label{r-implementation}

\chapter{Rcpp implementation}\label{rcpp-implementation}




\end{document}
